\documentclass[authoryearcitations]{UoYCSproject}
\author{David M. Taylor}
\title{Road Safety Advisory System}
\date{Version 0.1, 2014-November-9}
\supervisor{Dr. Radu Calinescu}
\BEng
\wordcount{xxx}


\abstract{ ... }


\acknowledgements{ ... }

\begin{document}
\maketitle
\listoffigures
\listoftables
\renewcommand*{\lstlistlistingname}{List of Listings}
\lstlistoflistings

\cleardoublepage
\label{sec:start}
\thispagestyle{empty}\cleardoublepage

\chapter{Introduction}

\section{Project Area and Motivation}

It is envisaged that open data, i.e. data that can be freely used and redistributed by anyone, released by governments will lead to major social and economic advances. Deloitte, one of the largest professional services firms in the world, believes that every business should have a strategy to exploit the growing estate of open data \citep{DeloitteAnalytics2012}. Furthermore, in the Open Data Charter of 18 June 2013, it is acknowledged that the use of open data can spur economic growth \citep{CabinetOffice2013}. The Open Data charter states that members of G8 are committed to releasing open data in order to create more accountable, responsive, and effective governments and businesses. Open data increases transparency about what government and businesses are doing, which promotes accountability and good governance.

The Open Data Charter also states that "freely-available government data can be used in innovative ways to create useful tools and products that help people navigate modern life more easily". It is this use of open data that forms the motivation for this project. There are already a large number of applications available that make use of various open data sets. The \textit{data.gov.uk} website has a catalogue of 350+ applications \citep{Data.go}, each of which take open datasets and make the data a usable resource for the general public. 

In this project, I will be looking at web applications with interactive maps, that make use of open data. There are several applications available in this area, but there isn't an agreed approach on how to develop such web applications. I will analyse the literature in this area, any existing applications, and the development options that are available. This will allow me to establish a suitable approach, before making my own application.

%something about road data

\section{Project Aims and Objectives}

The principle aim of this project is to develop and evaluate an application that allows a user to view and interact with open data via an interactive map. More specifically, I aim to build a web application that uses road safety data provided by the UK Department for Transport, in order to warn road users about accident hotspots.

I'm also aiming to build a framework that can be followed by other developers in the future, in order to simplify and encourage the development of similar applications. This will hopefully lead to more open data sets being used for the benefit of the general public.

For the planning phase of the project, I followed the steps outlined in Dawson's Projects in Computing and Information Systems \citep{Dawson2009}. As part of this process, I have identified the following high-level objectives:
\begin{enumerate}
	\item Complete a literature search and review of web development options, software engineering methodologies, open data paradigm, similar applications and road safety studies.
	\item Assemble a list of SMART requirements for the project.
	\item Develop the application on a suitable platform.
	\item Evaluate technology and techniques used.
	\item Complete final report.
\end{enumerate}

These objectives have been broken down into more specific objectives, which are shown in the work breakdown structure in Appendix A. In addition to this, I created a Gantt chart in order to plan the delivery of each objective within the time constraints of the project. The Gantt chart can also be seen in Appendix A. 


\section{Statement of Ethics}

\section{Report Structure}


\chapter{Literature Review}

\section{Open Data}

The formal definition of open data states that "open data is data that can be freely used, modified, and shared by anyone for any purpose" \citep{OpenKnowledge}. The release of open datasets by governments is becoming increasingly popular as the benefits become more apparent.

In this section of the report, I will look at the history of the open data movement, benefits and challenges that it introduces, available open datasets, applications using road safety data, and research areas involving open data.

\subsection{The Open Data Movement}

The origins of the open data movement can be traced back as far as 1942, when Robert King Merton explained the importance of making research results freely accessible to all \citep{Chignard2013}. The term open data wasn't coined until 1995, when an American scientific agency used it in reference to a complete and open exchange of scientific information between countries. Researchers were the first to recognise the benefits of sharing data in this manner. This concept of sharing data, along with the growing popularity of open source software led to the foundation of open data.

The open data movement in the UK started to pick up momentum as early as 2006, when The Guardian launched its "Free Our Data" campaign \citep{GuardianTechnology2006}. This campaign called for raw data gathered by Ordnance Survey to be made freely available for reuse by individuals and companies. This campaign partially achieved its goal when, on 1 April 2010, Ordnance Survey released the brand OS OpenData \citep{OrdnanceSurveyteam2010}. The campaign has since continued, with the aim of making more data publicly available. 

In 2007, a meeting was held in Sebastopol, USA, with the aim of defining the concept of open public data and have it adopted by the US presidential candidates. This meeting was hugely successful, and ultimately led to Barack Obama signing two presidential memoranda concerning Open Data the following year. 

In 2009, \textit{data.gov} was launched, with the objective of increasing public access to high value, machine readable datasets. Later in the year, the White House issued an Open Government Directive requiring federal agencies to take immediate, specific steps to achieve key milestones in transparency, participation, and collaboration \citep{Orszag2009}. This required that all agencies post at least three high-value data sets online and register them on \textit{data.gov}.

In 2009, the UK Prime Minister appointed the founder of the World Wide Web, Sir Tim Berners-Lee, as expert advisor on public information delivery \citep{CabinetOffice2009}. Berners-Lee was asked to oversee the work to create a single point of access for government held public data and develop proposals to extend access to data from the wider public sector, including selecting and implementing common standards. This work led to the launch of \textit{data.gov.uk} in January 2010. On launch the site contained 2,500 datasets, and it now contains over 19,500 from areas as diverse as transport, health, and the economy. This increase demonstrates the open data push that has taken place over the last few years in the UK, as the government strives to improve its openness.

On 9 May 2013, Barack Obama signed an executive order \citep{TheWhiteHouse-OfficeofthePressSecretary2013} that made open and machine-readable data the new default for government information. The US government has launched a number of open data initiatives aimed at scaling up open data efforts across the Health, Energy, Climate, Education, Finance, Public Safety, and Global Development sectors. The number of available datasets on \textit{data.gov} has grown rapidly in recent years, with over 130,000 now available. This shows that, like the UK, the US administration has taken large steps to improve its openness.


%Open Gov Partnership

%http://www.opengovpartnership.org/

%The Open Data Institute
%The Open Data Institute (ODI) was founded by Berners-Lee and Nigel Shadbolt in 2012. The ODI is a not-for-profit organisation, dedicated to promoting Open Data. The ODI aims to "catalyse the evolution of open data culture to create economic, environmental, and social value" [10]. In only its first year, the ODI made a big impact on the Open Data movement...

\subsection{Benefits and Challenges of Using Open Data}


%Tim O'Reilly played a big role in the Sebastopol meeting and the open data movement in the US as a whole. He has since written about how government can improve by embracing open data.  He believes that government can learn from open-source software, by becoming "an open platform that allows people inside and outside government to innovate" \citep{OReilly2011}. 

\subsection{Open Data Datasets}

\subsection{Uses of Road Traffic Data}

\subsection{Research in Open Data}

\section{Web Application Development}

\section{Software Development Methodologies}

\section{Road Traffic Studies}

\chapter{Requirement Analysis}

\section{Non-functional Requirements}

\begin{tabular}{| p{2.2cm} | p{7.5cm} | p{2cm} |}
	\hline
	\textbf{Requirement ID} & \textbf{Description} & \textbf{Importance} \\ \hline
	NFR1 & The results of any data analysis should be presented in a user-friendly way. & High \\ \hline
	NFR2 & The response time of the application should be <5 seconds, assuming that the user's connection isn't the bottleneck. & High \\ \hline
	NFR3 & The user interface should use standard controls. & Medium \\
	\hline
\end{tabular}

\section{Functional Requirements}

\begin{tabular}{| p{2.2cm} | p{7.5cm} | p{2cm} |}
	\hline
	\textbf{Requirement ID} & \textbf{Description} & \textbf{Importance} \\ \hline
	FR1 & The application should use road safety data published by the UK Department for Transport to warn users about accident hotspots. & High \\ \hline
	FR2 & Users should be able to specify road journeys, and the road safety data should be used to highlight risks on the selected route. & High \\ \hline
	FR3 & Users should be able to specify time of travel in order to further filter the data. & Medium \\ \hline
	FR4 & Users should be able to set warning thresholds, such as frequency and severity of accidents. & Medium \\ \hline
	FR5 & Users should be able to specify weather conditions in order to further filter the data. & Low \\ \hline
	FR6 & Weather data should be retrieved from the Met Office DataPoint service using the location and time parameters specified by the user. This data should then be used to filter the road safety data. & Low \\ \hline
	FR7 & The application should suggest lower risk routes for the specified journey. & Low \\ \hline
	FR8 & The location of incidents should be displayed on an interactive map. & High \\ \hline
	FR9 & Users should be able to retrieve additional information about an incident. & Medium \\ \hline
	FR10 & UThe application should be compatible with the latest versions of Google Chrome, Mozilla Firefox and Internet Explorer. & High \\ \hline
	FR11 & The application should be compatible with mobile devices. & Low \\ \hline
	FR12 & The site administrator should be able to add and remove datasets through a user-friendly GUI. & High \\ \hline
	FR13 & The road safety data should be automatically updated when new datasets are released. & Low \\ \hline
	FR14 & A generic ‘boilerplate’ implementation should be developed that can be reused to develop other Open Data applications. & Medium \\
	\hline
\end{tabular}

\chapter{Design}

\section{Development Options}
One of the principle objectives of this project is to find an elegant approach to building web applications that make use of open data. It is important to develop a framework that makes the development of such applications as simple as possible, in order to make it available to a wider audience. This will hopefully encourage other developers, regardless of experience in the field, to make web applications that use open data. 

One of the largest obstacles in creating an application like this is managing the different datasets. Hence, I believe that it’s important to create an interface for doing this, so that datasets can be added and removed with ease. It is also important that data is stored in a consistent manner, so that various external components can interact with any dataset.

Given these requirements, I believe that using a content management system (CMS) such as Drupal, Joomla or WordPress would be a suitable solution. By developing a component for importing the data into the CMS infrastructure, I will then be able to take advantage of the functionality that the CMS offers. This will enable me to use various components for interacting with the data, without having to develop everything from square one. This will not only benefit myself, but also future developers looking to work with open data. 

Using a CMS will enable me to create a dynamic site, which makes adding new features and functions far easier. Many content management systems are highly scalable, which is very important in projects like this, where a large amount of data is involved. Most popular CMSs have a large community who develop components for everyone to use, as well as providing support. In addition to all of this, a larger number of developers are already experienced in using CMSs, so they will be able to pick up my framework with ease. 

I will now evaluate some of the available content management systems, in order to decide which is the most suitable for this project.



\bibliography{library}


\end{document}
