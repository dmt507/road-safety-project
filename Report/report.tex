\documentclass[authoryearcitations]{UoYCSproject}
\author{David M. Taylor}
\title{Road Safety Advisory System}
\date{Version 0.1, 2014-November-9}
\supervisor{Dr. Radu Calinescu}
\BEng
\wordcount{xxx}


\abstract{ ... }


\acknowledgements{ ... }

\begin{document}
\maketitle
\listoffigures
\listoftables
\renewcommand*{\lstlistlistingname}{List of Listings}
\lstlistoflistings

\cleardoublepage
\label{sec:start}
\thispagestyle{empty}\cleardoublepage

\chapter{Introduction}
\label{cha:Introduction}

It is envisaged that open data 

The importance of this data is growing, as acknowledged by the G8 leaders, who signed the Open Data Charter on 18 June 2013 [2]. The charter states that members of G8 are committed to releasing open data in order to create more accountable, responsive, and effective governments and businesses, as well as spurring economic growth. Open data increases transparency about what government and businesses are doing, which promotes accountability and good governance. The charter also states that “freely-available government data can be used in innovative ways to create useful tools and products that help people navigate modern life more easily”. It is this use of open data that forms the motivation for this project. 

In this project, I will be looking at web applications with interactive maps that make use of open data. There are several applications available in this area, but there isn't an agreed approach on how to develop such web applications. I will analyse the literature in this area, any existing applications, and the development options that are available. This will allow me to establish a suitable approach, before making my own application. I aim to build a framework that can be followed by other developers in the future, in order to simplify and encourage the development of similar applications. This will hopefully lead to more open data sets being used for the benefit of the general public.
\citep{KopkaDaly1999}

\chapter{Literature Review}
\section{Open Data}
\subsection{The Open Data Movement}

The formal definition of open data states that “open data is data that can be freely used, modified, and shared by anyone for any purpose” [1].

The origins of the open data movement can be traced back as far as 1942, when Robert King Merton explained the importance of making research results freely accessible to all [3]. The term open data wasn’t coined until 1995, when an American scientific agency used it in reference to a complete and open exchange of scientific information between countries. Researchers were the first to recognise the benefits of sharing data in this manner. This concept of sharing data, along with the growing popularity of open source software led to the foundation of open data. In 2007, a meeting was held in the US, with the aim of defining the concept of open public data and have it adopted by the US presidential candidates. 

Tim O’Reilly has played a big role in the open data movement in the US. As well as being a key figure in the aforementioned 2007 meeting, he has since written about how government can improve by embracing open data.  He believes that government can learn from open-source software, by becoming “an open platform that allows people inside and outside government to innovate” [4]. 

In 2009,  the White House issued an Open Government Directive [5] requiring federal agencies to take immediate, specific steps to achieve key milestones in transparency, participation, and collaboration. On 9 May 2013, Barack Obama signed an executive order [6] that made open and machine-readable data the new default for government information. The US government has launched a number of open data initiatives aimed at scaling up open data efforts across the Health, Energy, Climate, Education, Finance, Public Safety, and Global Development sectors.

The open data movement in the UK started to pick up momentum as early as 2006, when The Guardian launched its “Free Our Data” campaign [7]. This campaign called for raw data gathered by Ordnance Survey to be made freely available for reuse by individuals and companies. This campaign partially achieved its goal when, on 1 April 2010, Ordnance Survey released the brand OS OpenData [8]. The campaign has since continued, with the aim of making more data publicly available. 

In 2009, the Prime Minister appointed the founder of the World Wide Web, Sir Tim Berners-Lee as expert advisor on public information delivery [9]. Berners-Lee was asked to oversee the work to create a single point of access for government held public data and develop proposals to extend access to data from the wider public sector, including selecting and implementing common standards. This work led to the launch of data.gov.uk in January 2010. On launch the site contained 2,500 datasets, and it now contains over 19,500. This increase demonstrates the Open Data push that has taken place over the last few years in the UK, as the government strives to improve its openness. 

Open Gov Partnership

http://www.opengovpartnership.org/

The Open Data Institute
The Open Data Institute (ODI) was founded by Berners-Lee and Nigel Shadbolt in 2012. The ODI is a not-for-profit organisation, dedicated to promoting Open Data. The ODI aims to "catalyse the evolution of open data culture to create economic, environmental, and social value" [10]. In only its first year, the ODI made a big impact on the Open Data movement...

\subsection{Benefits and Challenges of Using Open Data}

\subsection{Open Data Datasets}

\subsection{Uses of Road Traffic Data}

\subsection{Research in Open Data}

\section{Web Application Development}

\section{Software Development Methodologies}

\section{Road Traffic Studies}

\chapter{Requirement Analysis}

\section{Non-functional Requirements}

\section{Functional Requirements}

\chapter{Design}

\section{Development Options}




\bibliography{references}


\end{document}
